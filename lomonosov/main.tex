\documentclass{beamer}

% Russian-specific packages
%--------------------------------------
\usepackage[T2A]{fontenc}
\usepackage[utf8]{inputenc}
\usepackage[russian]{babel}

\hypersetup{
    unicode=true % otherwise get "Glyph not defined"
}
%--------------------------------------

% Asymptote for pictures
%--------------------------------------
\usepackage{asymptote} %% comes with options inline and attach
%--------------------------------------

\usetheme{default}

\newtheorem{stmt}{Утверждение}
\newtheorem{prblm}{Затруднение}

\title{Движение симметричного экипажа \\с массивными роликами на омни-колесах}

\author{К.Герасимов, А.А. Зобова,  И.И.Косенко}

\institute[Universities of Somewhere and Elsewhere]
{
  Кафедра теоретической механики и мехатроники\\
  МГУ им. М.В. Ломоносова
}

\date{Ломоносовские чтения, 2017}

% Delete this, if you do not want the table of contents to pop up at
% the beginning of each subsection:
\AtBeginSubsection[]
{
  \begin{frame}<beamer>{План}
    \tableofcontents[currentsection,currentsubsection]
  \end{frame}
}

% Let's get started
\begin{document}

\begin{frame}
  \titlepage
\end{frame}

\begin{frame}{План}
  \tableofcontents
  % You might wish to add the option [pausesections]
\end{frame}


\section{Постановка задачи}

\begin{frame}{Постановка задачи}{Рисунки}
    \begin{figure}
        \centering
        \minipage{0.5\textwidth}
            \asyinclude{pic_cart.asy}
            \caption{Экипаж}
        \endminipage
        \minipage{0.5\textwidth}
            \asyinclude{pic_wheel.asy}
            \caption{Колесо}
        \endminipage
        % \end{subfigure}
    \end{figure}
\end{frame}

\begin{frame}{Постановка задачи}{Тела, связи, степени свободы}
  \begin{itemize}
  \item {
    Экипаж состоит из платформы, $N$ колес и $n$ роликов,\\
    количество твердых тел:
    $$1 + N(n+1)$$
  }
  \item{
    Оси и центры колес и роликов неподвижны относительно\\
    платформы и колес соответственно
  }
  \item {
    Скорость точек контакта равна нулю:
    $$\vec{v}_{C_i} = 0, i = 1..3$$
  }
  \item{
    Количество степеней свободы:
    $$3 + N(n-1)$$
  }

  \end{itemize}
\end{frame}

\begin{frame}{Постановка задачи}{Координаты, псевдоскорости, связи}
  \begin{itemize}
  \item {
    Обобщенные координаты: \\
    $q = (x, y, \theta, \chi_i, \phi_k, \phi_s),$ где $i,k = 1..N, s$ -- остальные.
  }
  \item{
    Псевдоскорости:\\
    $\nu = (\nu_1, \nu_2, \nu_3, \nu_s), \vec{v_S} = R\nu_1\vec{e_\xi} + R\nu_2\vec{e_\eta}, \nu_3 = \Lambda\dot{\theta}, \nu_s = \dot{\phi_s}$
  }
  \item {
    Связи:
	$$ \dot{x} = \nu_1cos(\theta)R-\nu_2sin(\theta)R, \hspace{15pt} \dot{y} = \nu_1sin(\theta)R+\nu_2cos(\theta)R,$$
	$$\dot{\theta} = \frac{\nu_3}{\Lambda}, \hspace{15pt} \dot{\chi_i} = \frac{sin(\alpha_i)\nu_1R}{l} - \frac{cos(\alpha_i)\nu_2R}{l} - \frac{\nu_3R}{l\Lambda}, $$
	$$ \dot{\phi_k} = \frac{cos(\alpha_1)\nu_1R}{cos(chi1)l-r} + \frac{sin(\alpha_1)\nu_2R}{cos(chi1)l-r}, \hspace{15pt} \dot{\phi_s} = \nu_s $$
  }

  \end{itemize}
\end{frame}

\section{Уравнения движения}

\subsection{Кинетическая энергия и лагранжиан}

\begin{frame}{Кинетическая энергия и лагранжиан}
  \begin{itemize}
  \item {
    Присутствует аддитивный член, пропорциональный $B$ -- моменту инерции ролика относительно его оси собственного вращения:
    $$ 2T = 2L = M\vec{v_S}^2 + I_S\dot{\theta}^2 + J\sum_i\dot{\chi_i}^2 + $$
    $$ + \alert{B\sum_{i,j}(\dot{\phi_{ij}}^2 + 2\dot{\theta}sin(\kappa_j + \chi_i)\dot{\phi_{ij}})}, $$
    где
    $$ M = \mathring{M} + 3\cdot4m, $$
    $$ I_S = \mathring{I_S} + 3\cdot4(\frac{A+B}{2} + mR^2 + \frac{mr^2}{2}), $$
    $$ J = \mathring{J} + 4(A + mr^2) $$
  }

  \end{itemize}
\end{frame}

\begin{frame}{Кинетическая энергия и лагранжиан}
  \begin{itemize}
  \item {
    С учетом связей:
    $$ 2L^{*} = \mathring{\nu}^T \mathring{V}^T \mathring{M} \mathring{V} \mathring{\nu} + $$
    $$ + \alert{B}\sum_{i}(
    	\frac{(\nu_2sin(\alpha_i)+\nu_1cos(\alpha_i))^2R^2}
    	{\rho_i^2} + $$
    $$ +
    	\frac{2\nu_3(\nu_2sin(\alpha_i)+\nu_1cos(\alpha_i))sin(\chi_i)R}
    	{\rho_i\Lambda}
    ) $$
    $$ +
    \alert{B}\sum_{i,j}(
    	\frac{2\nu_3\nu_{ni+j}sin((\kappa_j+\chi_i)}
    	{\Lambda}
    	+
    	\nu_{ni+j}^2
    )
    $$
    где $ \frac{1}{2}\mathring{\nu}^T \mathring{V}^T \mathring{M} \mathring{V} \mathring{\nu} $ -- лагранжиан системы без роликов.
  }

  \end{itemize}
\end{frame}


\subsection{Структура уравнений - отличие от случая без роликов}

\begin{frame}{Структура уравнений}{Отличие от случая без роликов}
  \begin{itemize}
  \item {
    Уравнения Я.В. Татаринова:
    $$ \frac{d}{dt}\frac{\partial L^{*}}{\partial \nu_\alpha} = - \{P_\alpha, L^{*}\} + \{P_\alpha, \nu_\mu P_\mu\}, $$
    $$ \nu_\mu P_\mu = \dot{q_i} p_i, \hspace{10pt} p_i = \frac{\partial L}{\partial q_i} $$
  }
  \item {
    Лагранжиан и "импульсы" отличаются аддитивными членами:
    $$ L^{*} = \mathring{L}^{*} + BL^{*}_\Delta(\nu, \chi) $$
    $$ P_\alpha = \mathring{P_\alpha}(\theta, p_x, p_y, p_\chi) + P_\Delta(p_{\phi_i}, \chi) $$
  }

  \end{itemize}
\end{frame}

\begin{frame}{Структура уравнений}{Отличие от случая без роликов}
  $$ \frac{d}{dt}\frac{\partial L^{*}}{\partial \nu_\alpha} = - \{P_\alpha, L^{*}\} + \{P_\alpha, \nu_\mu P_\mu\}, $$
  \begin{itemize}
  \item {
    \begin{stmt}
    Уравнения движения отличаются на добавку порядка $B$.
    \end{stmt}
    $$ \frac{d}{dt}\frac{\partial }{\partial \nu_\alpha}(L^{*} - \mathring{L^{*}}) = \alert{B}\frac{d}{dt}\frac{\partial}{\partial \nu_\alpha}L^{*}_\Delta(\nu, \chi) $$
    $$ \{P_\alpha, L^{*}\} - \{\mathring{P}_\alpha, \mathring{L}^{*}\} = \alert{B}\{ P_\alpha, L^{*}_\Delta(\nu, \chi) \} $$
    $$ \{P_\alpha, P_\mu\} - \{\mathring{P}_\alpha, \mathring{P}_\mu\}  = $$
    $$ = \alert{B}\frac{R^2}{\Lambda}\sum_i\frac{f_\alpha(\nu, \chi)}{\rho^2_i}(\frac{R}{\rho_i}(\nu_1cos(\alpha_i) + \nu_2sin(\alpha_i)) + \frac{sin(\chi_i)\nu_3}{\Lambda})$$
  }
  \end{itemize}
\end{frame}

\section{Численное решение}

\subsection{Переход между роликами}

\begin{frame}{Переход между роликами}{Сложности и допущения}
  \begin{itemize}
  \item {
    \begin{prblm}[1]\label{p1}
    Уравнения вырождаются на стыках роликов.
    \end{prblm}
    Пусть переход на следующий ролик будет раньше стыка.
  }
  \item {
    \begin{prblm}[2]\label{p2}
    Ролики входят и выходят из состояния контакта.
    \end{prblm}
    Будем рассматривать только движение роликов, находящихся в контакте. При переходе сохраним значения $\nu_i$, а остальные величины пересчитаем по уравнениям связей.
  }
  \item {
    \begin{block}{Физическая аналогия}
    Инерция роликов мала, и движение при смене контакта быстро устанавливается.
    \end{block}
  }
  \end{itemize}
\end{frame}


\subsection{Примеры}

\subsection{Сравнение со случаем без роликов}


% Placing a * after \section means it will not show in the
% outline or table of contents.
\section*{Summary}

\begin{frame}{Summary}
  \begin{itemize}
  \item
    The \alert{first main message} of your talk in one or two lines.
  \item
    The \alert{second main message} of your talk in one or two lines.
  \item
    Perhaps a \alert{third message}, but not more than that.
  \end{itemize}
  
  \begin{itemize}
  \item
    Outlook
    \begin{itemize}
    \item
      Something you haven't solved.
    \item
      Something else you haven't solved.
    \end{itemize}
  \end{itemize}
\end{frame}

\begin{frame}{Blocks}
\begin{block}{Block Title}
You can also highlight sections of your presentation in a block, with it's own title
\end{block}
\begin{theorem}
There are separate environments for theorems, examples, definitions and proofs.
\end{theorem}
\begin{example}
Here is an example of an example block.
\end{example}
\end{frame}




\end{document}


